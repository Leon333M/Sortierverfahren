% Bachelorarbeit LaTeX Template
% Kompatibel mit VS Code + LaTeX Workshop
\documentclass[a4paper,12pt]{article}

% ----------------------
% Pakete
% ----------------------
\usepackage[utf8]{inputenc}
\usepackage[T1]{fontenc}
\usepackage[ngerman]{babel}
\usepackage{graphicx}
\usepackage{hyperref}
\usepackage{amsmath, amssymb}
\usepackage{caption}
\usepackage{subcaption}
\usepackage{geometry}
\usepackage{listings}
\usepackage{float}

% ----------------------
% Seitenränder
% ----------------------
\geometry{
    a4paper,
    left=30mm,
    right=30mm,
    top=25mm,
    bottom=25mm
}

% ----------------------
% C++ Code Settings
% ----------------------
\lstset{
    language=C++,
    basicstyle=\ttfamily\small,
    keywordstyle=\color{blue},
    commentstyle=\color{gray},
    stringstyle=\color{red},
    numbers=left,
    numberstyle=\tiny,
    stepnumber=1,
    numbersep=5pt,
    frame=single,
    breaklines=true,
    showstringspaces=false
}

% ----------------------
% Dokumentbeginn
% ----------------------
\begin{document}

% Titelseite
\begin{titlepage}
    \centering
    \vspace*{4cm}
    {\LARGE\textbf{Untersuchung der Skalierbarkeit von parallelem Sortieren auf einem Multicore-Prozessor}}\\[1.5cm]
    {\large Bachelorarbeit}\\[0.5cm]
    {\large Studiengang: Informatik}\\[0.5cm]
    {\large Bearbeiter: Leon Zoerner}\\[2cm]
    \vfill
    \today
\end{titlepage}

% Inhaltsverzeichnis / Gliederung
\tableofcontents
\newpage

% ----------------------
% 1. Einleitung
% ----------------------
\section{Einleitung}

\subsection{Motivation}
% Hier Motivation einfügen
Ziel dieser Arbeit ist es, die Grenzen von Threads und Parallelisierung aufzuzeigen. Dabei soll insbesondere untersucht werden, wie groß der Overhead durch Threads ist und welchen Performanceunterschied es macht, bereits initialisierte Workerthreads zu verwenden, im Vergleich zur Erstellung neuer Threads.
Da sich für diese Untersuchungen ein geeigneter, leicht verständlicher und programmierbarer Anwendungsfall anbietet, habe ich mich für Sortieralgorithmen entschieden, die sich zudem sehr gut parallelisieren lassen.

\subsection{Zielsetzung und Forschungsfrage}
% Hier Zielsetzung und Forschungsfrage einfügen
Ziel dieser Bachelorarbeit ist die systematische Analyse der Laufzeitentwicklung paralleler Sortierverfahren. Dabei soll untersucht werden, wie sich parallele Implementierungen von Quicksort und Mergesort im Vergleich zu ihren sequentiellen Varianten verhalten.
Im Fokus stehen insbesondere folgende Punkte:
\begin{itemize}
    \item der Einfluss verschiedener Threadingstrategien auf die Laufzeit,
    \item die Frage, ab welcher Eingangsgröße und bei welcher Anzahl von Threads ein messbarer Geschwindigkeitsvorteil entsteht,
    \item sowie die Identifikation von Thread-Management-Techniken, die für Sortieralgorithmen die besten Laufzeiten erzielen.
\end{itemize}
Aus diesen Aspekten ergibt sich die zentrale Forschungsfrage dieser Arbeit:
\newline
\textbf{Unter welchen Bedingungen liefern parallele Sortieralgorithmen anhand von Quicksort und Mergesort einen signifikanten Laufzeitvorteil gegenüber der sequentiellen Ausführung, und welche Threadingstrategien führen dabei zur besten Laufzeit?}


% ----------------------
% 2. Theoretische Grundlagen
% ----------------------
\newpage
\section{Theoretische Grundlagen}
\subsection{Sortieralgorithmen: Quicksort und Mergesort}
% Theorie zu Quicksort und Merge Sort
\subsection{Grundlagen der Parallelisierung}
% Amdahl Law, Overheads, Threading
\subsection{Thread-Modelle, Overheads und Skalierungsgrenzen}
% Thread-Erzeugung, Synchronisation, Grenzen der Parallelisierung

% ----------------------
% 3. Methodik und Versuchsaufbau
% ----------------------
\newpage
\section{Methodik und Versuchsaufbau}
\subsection{Messumgebung und Hardware}
% CPU, RAM, OS, Compiler

\subsection{Implementierungsvarianten}
% Sequenziell, parallel, Workerthreads, neue Threads

\subsection{Messmethodik}
% Vorgehensweise zur Laufzeitmessung, Datenerhebung

% Definiere Variablen
\newcommand{\Messziel}{Messziel}
\newcommand{\Erwartung}{Erwartung}
\newcommand{\Diagramm}{Diagramm}
\newcommand{\Analyse}{Analyse und Interpretation}

% ----------------------
% 4. Ergebnisse und Analyse
% ----------------------
\newpage
\section{Ergebnisse und Analyse}

\subsection{Grundlegende Laufzeiten abhängig von der Arraygröße}
\subsubsection{Messziel} % Einleitung
\subsubsection{Erwartung}
\subsubsection{Diagramm}
\subsubsection{Analyse und Interpretation}

\newpage
\subsection{Vergleich der Threading-Methoden}
\subsubsection{Messziel} % Einleitung
\subsubsection{Overheads und Skalierungsgrenzen}
\subsubsection{Erwartung}
\subsubsection{Diagramm}
\subsubsection{Analyse und Interpretation}

\newpage
\subsection{Einfluss der Arraygröße im Detail}
\subsubsection{Messziel} % Einleitung
\subsubsection{Erwartung}
\subsubsection{Diagramm}
\subsubsection{Analyse und Interpretation}

\newpage
\subsection{Einfluss des Listentyps}
\subsubsection{Messziel} % Einleitung
\subsubsection{Erwartung}
\subsubsection{Diagramm}
\subsubsection{Analyse und Interpretation}

\newpage
\subsection{Einfluss des Listenart}
\subsubsection{Messziel} % Einleitung
\subsubsection{Erwartung}
\subsubsection{Diagramm}
\subsubsection{Analyse und Interpretation}



% ----------------------
% 5. Diskussion und Fazit
% ----------------------
\newpage
\section{Diskussion und Fazit}
\subsection{Interpretation aller Ergebnisse}
% Ergebnisse interpretieren
\subsection{Beantwortung der Forschungsfrage}
% Forschungsfrage beantworten
\subsection{Zusammenfassung}
% Kurze Zusammenfassung der Arbeit

% ----------------------
% 6. Anhang
% ----------------------
\newpage
\section{Anhang}
\subsection{Code}
% Link oder Verweis auf Repository
Der gesamte Quellcode dieser Arbeit ist öffentlich unter folgendem Link verfügbar:
\newline
\href{https://github.com/Leon333M/Sortierverfahren}{https://github.com/Leon333M/Sortierverfahren}

\subsection{Hardware-Spezifikationen}
% Hardware genau dokumentieren
Zur besseren Einordnung der Leistungsfähigkeit der verwendeten Hardware befindet sich unter folgendem Link ein Benchmark:
\newline
\href{https://www.userbenchmark.com/UserRun/70984567}{https://www.userbenchmark.com/UserRun/70984567}
\newline
Ich liste aber jetzt hier auch nochmal die relevanten Hardwarekomponenten auf.
\newline
\href{https://geizhals.de/amd-ryzen-7-5800x-100-100000063wof-a2392525.html?t=alle&plz=&va=b&vl=de&hloc=de&v=e&togglecountry=set}
{CPU: AMD Ryzen 7 5800X, 8C/16T, 3.80-4.70GHz}
\newline
\href{https://geizhals.de/be-quiet-dark-rock-pro-4-bk022-a1794846.html?hloc=de}
{CPU Kühler: be quiet! Dark Rock Pro 4)}
\newline
\href{https://geizhals.de/g-skill-aegis-dimm-kit-16gb-f4-3200c16d-16gis-a2151626.html?hloc=de}
{RAM: G.Skill Aegis UDIMM 16GB Kit, DDR4-3200, CL16-18-18-38 (2 Kits, insgesamt 4x8 GB = 32 GB)}
\newline
Betriebssystem: Windows 10 Version 22H2

% ----------------------
% Dokumentende
% ----------------------
\end{document}
