% Bachelorarbeit LaTeX Template
% Kompatibel mit VS Code + LaTeX Workshop
\documentclass[a4paper,12pt]{article}

% ----------------------
% Pakete
% ----------------------
\usepackage[utf8]{inputenc}
\usepackage[T1]{fontenc}
\usepackage[ngerman]{babel}
\usepackage{graphicx}
\usepackage{hyperref}
\usepackage{amsmath, amssymb}
\usepackage{caption}
\usepackage{subcaption}
\usepackage{geometry}
\usepackage{listings}
\usepackage{float}

% ----------------------
% Seitenränder
% ----------------------
\geometry{
    a4paper,
    left=30mm,
    right=30mm,
    top=25mm,
    bottom=25mm
}

% ----------------------
% C++ Code Settings
% ----------------------
\lstset{
    language=C++,
    basicstyle=\ttfamily\small,
    keywordstyle=\color{blue},
    commentstyle=\color{gray},
    stringstyle=\color{red},
    numbers=left,
    numberstyle=\tiny,
    stepnumber=1,
    numbersep=5pt,
    frame=single,
    breaklines=true,
    showstringspaces=false
}

% ----------------------
% Dokumentbeginn
% ----------------------
\begin{document}

% Titelseite
\begin{titlepage}
    \centering
    \vspace*{4cm}
    {\LARGE\textbf{Untersuchung der Skalierbarkeit von parallelem Sortieren auf einem Multicore-Prozessor}}\\[1.5cm]
    {\large Bachelorarbeit}\\[0.5cm]
    {\large Studiengang: Informatik}\\[0.5cm]
    {\large Bearbeiter: Leon Zoerner}\\[2cm]
    \vfill
    \today
\end{titlepage}

% Inhaltsverzeichnis
\tableofcontents
\newpage

% ----------------------
% 1. Einleitung
% ----------------------
\section{Einleitung}
\subsection{Motivation}
% Hier Motivation einfügen

\subsection{Zielsetzung und Forschungsfrage}
% Hier Zielsetzung und Forschungsfrage einfügen

% ----------------------
% 2. Theoretische Grundlagen
% ----------------------
\section{Theoretische Grundlagen}
\subsection{Sortieralgorithmen: Quicksort \& Merge Sort}
% Theorie zu Quicksort und Merge Sort

\subsection{Grundlagen der Parallelisierung}
% Parallele Programmierung, Skalierungsgesetze, Overheads

\subsection{Thread-Modelle, Overheads und Skalierungsgrenzen}
% Thread-Erzeugung, Synchronisation, Grenzen der Parallelisierung

% ----------------------
% 3. Methodik und Versuchsaufbau
% ----------------------
\section{Methodik und Versuchsaufbau}
\subsection{Messumgebung und Hardware}
% Hardware und Umgebung beschreiben

\subsection{Testaufbau und Implementierungsvarianten}
% Quicksort/Merge Sort Varianten, Threading-Strategien

\subsection{Messmethodik}
% Vorgehensweise zur Laufzeitmessung, Datenerhebung

% ----------------------
% 4. Ergebnisse und Analyse
% ----------------------
\section{Ergebnisse und Analyse}
\subsection{Laufzeitmessungen}
% Messergebnisse darstellen

\subsection{Einfluss der Variablen }%(Threadanzahl, Arraygröße, Thread-Strategien)}
% Analysen zu Parametern

\subsection{Bewertung der Parallelisierungseffizienz}
% Effizienz-Analyse

%\subsection{Vergleich mit Python} % optional
% Optionaler Vergleich mit Python

\subsection{Grenzen und Fehlerbetrachtung}
% Diskussion von Limitationen

% ----------------------
% 5. Diskussion und Fazit
% ----------------------
\section{Diskussion und Fazit}
\subsection{Interpretation der Ergebnisse}
% Ergebnisse interpretieren

\subsection{Beantwortung der Forschungsfrage}
% Forschungsfrage beantworten

\subsection{Zusammenfassung}
% Kurze Zusammenfassung der Arbeit

% ----------------------
% 6. Anhang
% ----------------------
\section{Anhang}
\subsection{Code}
% Link oder Verweis auf Repository

\subsection{Exakte Hardware-Spezifikationen}
% Hardware genau dokumentieren

% ----------------------
% Dokumentende
% ----------------------
\end{document}
