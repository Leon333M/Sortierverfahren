\documentclass[10pt, ngerman]{beamer}

\usetheme{Madrid} 

% ----------------------
% Pakete
% ----------------------
\usepackage[utf8]{inputenc}
\usepackage[T1]{fontenc}
\usepackage[ngerman]{babel}
\usepackage{graphicx}
\usepackage{hyperref}
\usepackage{amsmath, amssymb}
\usepackage{caption}
\usepackage{subcaption}
\usepackage{geometry}
\usepackage{listings}
\usepackage{float}
\usepackage{mathtools} 
\usepackage{tikz}
\usepackage{pgfplots}
\usetikzlibrary{positioning, arrows.meta}
\pgfplotsset{compat=1.18}

\hypersetup{
    pdfborder={0 0 0},    % Entfernt alle Rahmen um Links
}

% ----------------------
% C++ Code Settings
% ----------------------
\lstset{
    language=C++,
    basicstyle=\ttfamily\small,
    keywordstyle=\color{blue},
    commentstyle=\color{gray},
    stringstyle=\color{red},
    numbers=left,
    numberstyle=\tiny,
    stepnumber=1,
    numbersep=5pt,
    frame=single,
    breaklines=true,
    showstringspaces=false
}

\title[Skalierbarkeit paralleles Sortieren]{Untersuchung der Skalierbarkeit von parallelem Sortieren auf einem Multicore-Prozessor}
\subtitle{Verteidigung der Bachelorarbeit}
\author{Leon Zoerner}
\institute{Informatik}
\date{\today}

% ----------------------
% Ausgelagerte Texte für die Bachelorarbeit
% ----------------------
% \input{commands.tex}
\newcommand{\WorkerThreadErklaerung}{
    Ein \emph{Worker-Thread} ist ein Thread, der einmalig erzeugt wird und über die gesamte Laufzeit des Programms aktiv bleibt.
    Seine Hauptaufgabe besteht darin, Aufgaben aus einer zugewiesenen Warteschlange kontinuierlich abzuarbeiten.
    Dies reduziert den Overhead durch ständiges Erzeugen und Zerstören von Threads und ermöglicht eine effiziente parallele Verarbeitung.
}

\newcommand{\WorkStealingErklaerung}{
    \emph{Work-Stealing} bezeichnet ein Verfahren, bei dem Aufgaben in prozessorspezifischen Warteschlangen verteilt werden und jeder Prozessor auf seiner lokalen Warteschlange operiert.
    Dabei können Prozessoren Aufgaben aus anderen Warteschlangen stehlen, um eine gleichmäßige Lastverteilung (\emph{Load Balancing}) zu erhalten \cite{dynamicscheduling}.
}

\newcommand{\WorkStealingAehnlichErklaerung}{
    Der \emph{Work-Stealing-ähnliche Ansatz} in dieser Arbeit basiert auf einem Worker-Thread-Pool mit einer gemeinsamen Aufgabenwarteschlange.
    Alle Worker-Threads ziehen Aufgaben aus dieser Warteschlange und bearbeiten sie.
    Dazu kann jeder Worker-Thread auch neue Aufgaben in die Warteschlange legen.
    Ist die Queue leer, warten die Threads auf neue Aufgaben.
    Dadurch wird sichergestellt, dass alle Threads möglichst konstant ausgelastet sind, ohne dass für jeden Thread eine eigene Warteschlange benötigt wird.
    Dies reduziert den Overhead durch Synchronisation und vermeidet Leerlaufzeiten, ähnlich wie beim klassischen Work-Stealing.
}

\newcommand{\WorkerthreadsBegriffsdefinition}{
    \WorkerThreadErklaerung

    \WorkStealingErklaerung
}

\newcommand{\StrongScalingBegriffsdefinition}{
    Unter \emph{Strong Scaling} versteht man die Untersuchung der Laufzeit eines festen Problemumfangs bei steigender Anzahl an Recheneinheiten (Threads).
    Ziel ist es zu analysieren, wie stark sich die Laufzeit durch zusätzliche Parallelisierung verkürzt.
}

\newcommand{\WeakScalingBegriffsdefinition}{
    Unter \emph{Weak Scaling} versteht man die Untersuchung der Laufzeit, bei der der Problemumfang proportional zur Anzahl der Recheneinheiten wächst.
    Ziel ist es zu bewerten, ob die Laufzeit bei wachsender Parallelität konstant bleibt.
}

\newcommand{\TiefenbasierteThreadErzeugungBegriffsdefinition}{
    \StrongScalingBegriffsdefinition

    \WeakScalingBegriffsdefinition
}

\newcommand{\OverheadBegriffsdefinition}{
    Unter \emph{Overhead} versteht man
    sämtliche hardwarebedingten Overheads, wie z.\,B. Speicherlatenzen, sowie
    sämtliche softwarebedingten Overheads, wie z.\,B. die Synchronisation von Threads.
}
% Laufzeit bei Arraygröße 16 = Thread-Overhead
% Laufzeit bei Arraygröße n = Strong Scaling
% Threads und Länge verdoppeln Weak Scaling (Start Arraygröße n ) = Weak Scaling
% Länge verdoppeln (16 Threads)

\newcommand{\InkrementArrayDiagrammA}{%
    \begin{tikzpicture}
        \begin{axis}[
                title style={yshift=1.5ex},
                width=1\textwidth,
                height=0.6\textwidth,
                xlabel={Thread-Anzahl},
                ylabel={Arraygröße},
                title={50\% Geschwindigkeitssteigerung}, % Zeitpunkt der 50\% Geschwindigkeitssteigerung (incAray)
                xmin=1, xmax=16,
                ymin=2^19, ymax=2^22,
                grid style=dashed,
                legend pos=north east,
                ymode=log,
                log basis y=2,
                xtick={1,...,16},               % jeden Integer von 2 bis 17
                ytick={2^20,2^21},      % gewünschte y-Werte
                xticklabels={$1$,$2$,$3$,$4$,$5$,$6$,$7$,$8$,$9$,$10$,$ $,$12$,$ $,$14$,$ $,$16$},
                grid=both,
                grid style=dashed,
            ]
            \addplot[only marks, blue, mark=*] coordinates {
                    (1,2)
                    (3,2097152)
                    (4,2097152)
                    (5,1048576)
                    (6,1048576)
                    (7,1048576)
                    (8,1048576)
                    (9,1048576)
                    (10,1048576)
                    (11,1048576)
                    (12,1048576)
                    (13,2097152)
                    (14,2097152)
                    (15,2097152)
                    (16,2097152)
                };
            \addlegendentry{Messwerte}
        \end{axis}
    \end{tikzpicture}%
}

\newcommand{\InkrementArrayDiagrammB}{%
    \begin{tikzpicture}
        \begin{axis}[
                title style={yshift=1.5ex},
                width=1\textwidth,
                height=0.6\textwidth,
                xlabel={Thread-Anzahl},
                ylabel={Dauer in ns},
                title={Strong Scaling: Laufzeit bei Arraygröße $2^{21}$},
                xmin=0, xmax=16,
                ymin=0.5*10^6, ymax=2*10^6,
                grid style=dashed,
                legend pos=north east,
                xtick={1,...,16},
                xticklabels={$1$,$2$,$3$,$4$,$5$,$6$,$7$,$8$,$9$,$10$,$ $,$12$,$ $,$14$,$ $,$16$},
                grid=both,
                grid style=dashed,
            ]
            \addplot[blue, mark=*] coordinates {
                    (1,1852600)
                    (2,1524000)
                    (3,1214400)
                    (4,975200)
                    (5,856500)
                    (6,795800)
                    (7,769500)
                    (8,809800)
                    (9,809400)
                    (10,775200)
                    (11,796400)
                    (12,802000)
                    (13,869400)
                    (14,829300)
                    (15,878800)
                    (16,910300)
                };
            \addlegendentry{Messwerte}
        \end{axis}
    \end{tikzpicture}%
}

\newcommand{\InkrementArrayDiagrammC}{%
    \begin{tikzpicture}
        \begin{axis}[
                title style={yshift=1.5ex},
                width=1\textwidth,
                height=0.6\textwidth,
                xlabel={Thread-Anzahl},
                ylabel={Dauer in ns},
                title={Strong Scaling: Laufzeit bei Arraygröße $2^{31}$}, % ($2^{31}-1$)
                xmin=0, xmax=16,
                ymin=0.5*10^9, ymax=0.2*10^10,
                grid style=dashed,
                legend pos=north east,
                xtick={1,...,16},
                xticklabels={$1$,$2$,$3$,$4$,$5$,$6$,$7$,$8$,$9$,$10$,$ $,$12$,$ $,$14$,$ $,$16$},
                grid=both,
                grid style=dashed,
            ]
            \addplot[blue, mark=*] coordinates {
                    (1,1582322800)
                    (2,1169109200)
                    (3,995920200)
                    (4,876230000)
                    (5,818310300)
                    (6,786585300)
                    (7,767543800)
                    (8,769668500)
                    (9,762244800)
                    (10,754572500)
                    (11,751793100)
                    (12,753548100)
                    (13,787192000)
                    (14,755368000)
                    (15,755257400)
                    (16,751066600)
                };
            \addlegendentry{Messwerte}
        \end{axis}
    \end{tikzpicture}%
}

\newcommand{\InkrementArrayDiagrammD}{%
    \begin{tikzpicture}
        \begin{axis}[
                title style={yshift=1.5ex},
                width=1\textwidth,
                height=0.6\textwidth,
                xlabel={Thread-Anzahl},
                ylabel={Dauer in ns},
                title={Laufzeit bei Arraygröße 16},
                grid=both,
                grid style=dashed,
                xmin=0, xmax=16,
                ymin=0, ymax=5*10^5,
                legend pos=north west,
                xtick={1,...,16},
                ytick={1*10^5,2*10^5,3*10^5,4*10^5,5*10^5},
                xticklabels={$1$,$2$,$3$,$4$,$5$,$6$,$7$,$8$,$9$,$10$,$ $,$12$,$ $,$14$,$ $,$16$},
            ]
            \addplot[blue, mark=*] coordinates {
                    (1,200)
                    (2,145900)
                    (3,131500)
                    (4,153200)
                    (5,171200)
                    (6,211400)
                    (7,248900)
                    (8,274200)
                    (9,252300)
                    (10,284400)
                    (11,330400)
                    (12,351600)
                    (13,397200)
                    (14,391100)
                    (15,415000)
                    (16,471400)
                };
            \addlegendentry{Messwerte}
        \end{axis}
    \end{tikzpicture}%
}

\newcommand{\InkrementArrayDiagrammE}{%
    \begin{tikzpicture}
        \begin{axis}[
                title style={yshift=1.5ex},
                width=1\textwidth,
                height=0.6\textwidth,
                xlabel={Thread-Anzahl},
                ylabel={Dauer in ns},
                title={Weak Scaling {\small (Start-Größe $2^{21}$)}}, % Threads und Länge verdoppeln
                xmin=0, xmax=16,
                ymin=1*10^6, ymax=1.5*10^7,
                grid style=dashed,
                legend pos=north west,
                xtick={1,2,4,8,16},
                ytick={1*10^6, 5*10^6, 10*10^6, 1.5*10^7},
                grid=both,
                grid style=dashed,
            ]
            \addplot[blue, mark=*] coordinates {
                    (1,1910700)
                    (2,2925400)
                    (4,3837500)
                    (8,6358100)
                    (16,12003500)
                };
            \addlegendentry{Messwerte}
        \end{axis}
    \end{tikzpicture}%
}

\newcommand{\InkrementArrayDiagrammF}{%
    \begin{tikzpicture}
        \begin{axis}[
                title style={yshift=1.5ex},
                width=1\textwidth,
                height=0.6\textwidth,
                xlabel={Arraygröße},
                ylabel={Dauer in ns},
                title={Arraygröße verdoppeln (16 Threads)},
                xmin=2^21, xmax=1 * 2^25,
                ymin=1*10^6, ymax=1.5*10^7,
                grid style=dashed,
                legend pos=north west,
                xtick={2^21,2^23,2^24,2^25},
                xticklabels={$2^{21}$, $2^{23}$, $2^{24}$, $2^{25}$},
                scaled x ticks=false,
                ytick={1*10^6, 5*10^6, 10*10^6, 1.5*10^7},
                grid=both,
                grid style=dashed,
            ]
            \addplot[blue, mark=*] coordinates {
                    (2097152,1077100)
                    (4194304,1675200)
                    (8388608,3237700)
                    (16777216,6213600)
                    (33554432,13184900)
                };
            \addlegendentry{Messwerte}
        \end{axis}
    \end{tikzpicture}%
}

\newcommand{\InkrementArrayText}{
    Anhand dieses einfachen Beispiels soll gezeigt werden, was Parallelisierung in der Praxis bewirkt und dass die Praxis nicht immer mit den Erwartungen übereinstimmt. Zudem stellt dieses Beispiel einen guten Einstieg in das Thema Parallelisierung dar.
    An den Diagrammen (Abbildung~\ref{fig:InkrementArrayDiagramm}) ist deutlich zu erkennen, dass dieses Beispiel nicht linear mit der Thread-Anzahl skaliert, obwohl dies rein theoretisch zu erwarten wäre. Dies liegt wahrscheinlich an der Speicherlatenz als limitierendem Faktor. Dies würde erklären, warum ab einem gewissen Punkt mehr ausgelastete Kerne keinen weiteren Performancegewinn mehr bringen. Zudem ist zu erkennen, dass das Array mindestens $2^{20}$ groß sein muss, damit eine parallele Ausführung einen mindestens zweifachen Geschwindigkeitsvorteil gegenüber der sequenziellen Laufzeit erreicht.
    Zusätzlich ist anzumerken, dass die sequenzielle Laufzeit dieser Funktion normalerweise unter 100~ns liegt. Dies ist auf Compiler-Optimierungen zurückzuführen. Daher wurde diese Funktion mit \texttt{volatile} ausgeführt, was verhindert, dass der Compiler die eigentliche Aufgabe herausoptimiert und sie somit messbar bleibt. Das \texttt{volatile}-Schlüsselwort sorgt dafür, dass bei jedem Lesevorgang die Daten aus dem RAM geladen werden müssen. Daher ist es auch das wahrscheinlichste Szenario, dass dieses Beispiel durch das Datenratenlimit begrenzt ist. Zusätzlich ist die eigentliche Aufgabe trivial für die CPU und lastet diese daher nicht vollständig aus.
    Aufgrund dieses künstlich erzeugten Szenarios durch die bewusste Beeinflussung des Laufzeitverhaltens mit \texttt{volatile} wird auf eine detaillierte Einzelanalyse der Diagramme verzichtet.
}



\newcommand{\InkrementArray}{
    % Inkrement-Array
    \InkrementArrayText
    \newline
    %\resizebox{\textwidth}{8cm}{%
    \begin{minipage}{\textwidth}
        \centering
        % Zeile 1
        \begin{minipage}{0.48\textwidth}
            \centering \InkrementArrayDiagrammA
            \captionof{figure}[Inkrement-Array (50\% Geschwindigkeitssteigerung)]{
                Zeigt, ab welcher Arraygröße zum ersten Mal eine 50\,\% höhere Geschwindigkeit als die sequentielle Laufzeit erreicht wird.
            }
        \end{minipage}\hfill %
        \begin{minipage}{0.48\textwidth}
            \centering \InkrementArrayDiagrammD
            \captionof{figure}[Inkrement-Array (Laufzeit bei Arraygröße 16)]{
                Zeigt die Overheads, die durch Threads entstehen.
            }
        \end{minipage}
        \par\vspace{1em}

        % Zeile 2
        \begin{minipage}{0.5\textwidth}
            \centering \InkrementArrayDiagrammB
            \captionof{figure}[Inkrement-Array (Laufzeit bei Arraygröße $2^{21}$)]{}
        \end{minipage}%
        \begin{minipage}{0.5\textwidth}
            \centering \InkrementArrayDiagrammC
            \captionof{figure}[Inkrement-Array (Laufzeit bei Arraygröße $2^{31}$)]{}
        \end{minipage}
        \par\vspace{1em}

        % Zeile 3
        \begin{minipage}{0.5\textwidth}
            \centering \InkrementArrayDiagrammE
            \captionof{figure}[Inkrement-Array (Weak Scaling {\small (Start-Größe $2^{21}$)})]{}
        \end{minipage}%
        \begin{minipage}{0.5\textwidth}
            \centering \InkrementArrayDiagrammF
            \captionof{figure}[Inkrement-Array (Arraygröße verdoppeln (16 Threads))]{}
        \end{minipage}

        \captionof{figure}[Laufzeitanalyse Inkrement-Array]{Laufzeitanalyse Inkrement-Array}
        \label{fig:InkrementArrayDiagramm}
    \end{minipage}
    %}
}
% Einleitung.tex

% Definiere Variablen
% \newcommand{\Messziel}{Messziel}

\newcommand{\Motivation}{
    Die Motivation dieser Arbeit war es, herauszufinden, wie sehr man mit Threads an die theoretisch erwartete Laufzeitverbesserung herankommen kann und welche Strategie dafür am besten geeignet ist.
    Da sich für diese Untersuchungen ein geeigneter, leicht verständlicher und programmierbarer Anwendungsfall anbietet, werden Sortieralgorithmen betrachtet, die sich zudem sehr gut parallelisieren lassen.
    Dazu kommt, dass ich Recherche nach Möglichkeit vermeiden wollte. Daher war es naheliegend, selbst Code zu schreiben und diesen zu analysieren, da man hierfür weniger recherchieren muss.
}

\newcommand{\ZielsetzungUndForschungsfrage}{
    % Hier Zielsetzung und Forschungsfrage einfügen
    Ziel dieser Bachelorarbeit ist die systematische Analyse der Laufzeitentwicklung paralleler Sortierverfahren. Dabei soll untersucht werden, wie sich parallele Implementierungen von Quicksort und Mergesort im Vergleich zu ihren sequentiellen Varianten verhalten.
    Im Fokus stehen insbesondere folgende Punkte:
    \begin{itemize}
        \item der Einfluss verschiedener Threadingstrategien auf die Laufzeit,
        \item die Frage, ab welcher Eingabegröße und bei welcher Anzahl von Threads ein messbarer Geschwindigkeitsvorteil entsteht,
        \item sowie die Identifikation von Thread-Management-Techniken, die für Sortieralgorithmen die besten Laufzeiten erzielen.
    \end{itemize}
    Aus diesen Aspekten ergibt sich die zentrale Forschungsfrage dieser Arbeit:
    \newline
    \textbf{Unter welchen Bedingungen liefern parallele Sortieralgorithmen anhand von Quicksort und Mergesort einen signifikanten Laufzeitvorteil gegenüber der sequentiellen Ausführung, und welche Threadingstrategien führen dabei zur besten Laufzeit?}
}

% Theoretische Grundlagen.tex

% Definiere Variablen
% \newcommand{\Messziel}{Messziel}

% ----------------------
% 2. Theoretische Grundlagen
% ----------------------
% \section{Theoretische Grundlagen}
% \subsection{Sortieralgorithmen: Quicksort und Mergesort}
% % Theorie zu Quicksort und Merge Sort
% \subsection{Grundlagen der Parallelisierung}
% % Amdahl Law
% \subsection{Thread-Modelle, Overheads und Skalierungsgrenzen}
% % Thread-Erzeugung, Synchronisation, Grenzen der Parallelisierung

% \subsection{Sortieralgorithmen: Quicksort und Mergesort}
% % Theorie zu Quicksort und Merge Sort
\newcommand{\SortieralgorithmenQuicksortMergesort}{
    Sowohl \textbf{Quicksort} als auch \textbf{Mergesort} basieren auf dem \emph{Teile-und-Herrsche}-Prinzip und sind rekursive Sortieralgorithmen. Dabei wird das zu sortierende Array wiederholt in kleinere Teilprobleme zerlegt, die unabhängig voneinander verarbeitet werden.

    \subsubsection{Mergesort}
    Das Grundprinzip von \textbf{Mergesort} besteht darin, aus zwei bereits sortierten Teillisten eine neue sortierte Liste zu erzeugen. In dieser Arbeit wird die unsortierte Ausgangsliste rekursiv in zwei möglichst gleich große Hälften geteilt, bis jede Teilliste nur noch aus einem einzelnen Element besteht. Da eine Liste mit einem Element per Definition sortiert ist, beginnt anschließend der sogenannte \emph{Merge-Schritt}. In diesem Schritt werden jeweils zwei sortierte Teillisten zu einer neuen sortierten Liste zusammengeführt.
    \newline
    Hierfür werden beide Teillisten sequenziell durchlaufen und die Elemente verglichen, wodurch pro Merge-Schritt \(n\) Vergleiche sowie \(2n\) Lese- und Schreibzugriffe erforderlich sind.
    \newline
    Die Laufzeit von Mergesort lässt sich durch die Rekurrenzgleichung
    \[
        T(n) = 2 \cdot T(n/2) + n
    \]
    beschreiben, wobei der Term \(+n\) den Aufwand des Merge-Schritts repräsentiert. Daraus ergibt sich eine Gesamtlaufzeit von
    \[
        T(n) = n \cdot \log_2(n)
    \]
    bzw.\ in asymptotischer Notation \(O(n \log n)\).

    \subsubsection{Quicksort}
    \textbf{Quicksort} ist im Grundaufbau ähnlich strukturiert, unterscheidet sich jedoch wesentlich im Ablauf. Die Liste wird nicht zwingend in zwei gleich große Hälften geteilt. Stattdessen wird zunächst ein sogenanntes \emph{Pivot-Element} gewählt, anhand dessen die Liste in einen kleineren und einen größeren Teil partitioniert wird. Dieser Partitionierungsschritt erfolgt \emph{vor} den rekursiven Selbstaufrufen, weshalb sich Quicksort auch als iterative Variante formulieren lässt.
    Beim Partitionieren wird die Liste so umsortiert, dass alle Elemente, die kleiner als das Pivotelement sind, links davon stehen und alle Elemente, die größer sind, rechts davon stehen. Dabei werden die Elemente auf beiden Seiten entsprechend getauscht.
    \newline
    Die Laufzeit von Quicksort hängt stark von der Qualität der Partitionierung ab. Im \textbf{Worst-Case}, beispielsweise bei ungünstiger Pivot-Wahl, beträgt die serielle Laufzeit
    \[
        T(n) = \frac{1}{2} \cdot ( n^2 + n ),
    \]
    \[
        O(T(n)) = n^2.
    \]
    Im \textbf{Best-Case} sowie im \textbf{Average-Case} ergibt sich hingegen ebenfalls die Rekurrenzgleichung
    \[
        T(n) = 2 \cdot T(n/2) + n,
    \]
    woraus wiederum eine Laufzeit von \(O(n \log n)\) folgt. Damit ist Quicksort im Durchschnitt asymptotisch genauso effizient wie Mergesort, aber in der Praxis ist Quicksort oft doppelt so schnell wie Mergesort, doch dazu später mehr.

}

% \subsection{Grundlagen der Parallelisierung}
% % Amdahl Law
\newcommand{\GrundlagenDerParallelisierung}{
    Parallelisierung beschreibt die gleichzeitige Ausführung mehrerer Programmteile mit dem Ziel, die Gesamtlaufzeit einer Berechnung zu reduzieren. Dabei wird eine ursprünglich serielle Aufgabe in mehrere Teilaufgaben zerlegt, die parallel auf mehreren Recheneinheiten verarbeitet werden können.
    \newline
    Der maximal erreichbare Geschwindigkeitsgewinn durch Parallelisierung ist jedoch begrenzt. Nach dem Amdahlschen Gesetz hängt die theoretische Beschleunigung davon ab, welcher Anteil eines Programms parallelisiert werden kann. Serielle Programmanteile sowie zusätzlicher Verwaltungsaufwand, beispielsweise durch Thread-Erzeugung, Synchronisation und Kommunikation, begrenzen die Skalierbarkeit.
    \newline
    In der Praxis führt Parallelisierung daher nicht zwangsläufig zu einer linearen Beschleunigung, insbesondere bei steigender Anzahl von Threads.
    \newline
    Einfach ausgedrückt bedeutet dies, dass Code mit seriellen Abhängigkeiten auch bei mehreren Threads nicht schneller ausgeführt wird. Daher sollte nur der Teil des Codes parallelisiert werden, der keine solchen Abhängigkeiten enthält.
}

% \subsection{Thread-Modelle, Overheads und Skalierungsgrenzen}
% % Thread-Erzeugung, Synchronisation, Grenzen der Parallelisierung
\newcommand{\ThreadModelleOverheadsUndSkalierungsgrenzen}{
    Bei Thrahds gibt es in der prxis immer Overhads die dafür sorgen das man eine perfekte Zeitersparnis nicht ereichen kann.
    Dise Overhads sind init-/join-zeiten, dekusturktor zeiten. Dazu komt das die Hardware auch in der Prxis genzen hatt was wider zu Overhads führt. Dise Hardware grenzen sind die thrad swaping zeiten, die begrantze kern anzahl (bei mir 8 kerne bzw 16 virtullle kerne), sowie kan es zu mer cach mises kommen wen man mehr thrads nutzt, da diser öfter benutzt wird. Dies führt widerum das mer Daten von Ram gelesen werden müssen was widerum dazu führen kann das man eher am Datenraten limit des ramcontolrers ist anstatt am rechenlimt der cpu kerne. Jetzt mus man auch bedekenden das man ziwcehen echten kerenen und virtuellen keren unterscheiden muss da ein virtuleer kern nicht so gut mit parelisung scalt wie ein echter aufgrund der begrenzeten hardware recurecen.

    Dann gibt es auch noch andre Skalierungsgrenzen wie die Listen göße, da wen eine lsite mer speicher bracuht als der ganze ram biten kan disen speicher auslgern muss, und dier ausgelgerter speicher ist wesenlich langsamer im lesen schreben sowe in den latzenen dafür.

    Dann gibt es natülich auch verschidene methoden wie man Code paraliseren kann. die einfachste ist den nicht sereien code immer in ein neuen thrad asuzulagern. Dise ist jedoch je nach anwedungs fall nicht immer sinnfoll, da man sonst zu vilee thrads erstellt die sich gegeseitetig duch thrads swoping oder gar duch den erhäteh ram verbrauch von den thrads selber gegseitig verlagsomern.
    Daher begrent man in normal fall die Thrad anzal um sowas zu vermeiden, das ist denke ich für anfänger die beste startegie.
    Dann kann man das genze natülich noch verbessern, indem man versucht die Thrad initzeiten nur eieinzigen mal für den geamten code vraucht, dise satatie gie wird dann mitels worker Trahds umgestetzt. Dise erstellt man einmal am anfang es codes und weisten ihnen imer wider neue aufgaben zu anstett wen sie fertig sind eichaf wider zu zerstören. Dann kan man das ganzie imer noch verändern in dem man die art und weise wie man diese workerrthads auszutz ändert, damit miene ich das man dise nur Aufgen zuwasien bracuht wen sie frei sind und anstat zu warten das sie frei werden im eigenen thrad direkt dise aufgbe zu erledigen. Dise startegie hat je nach anwedungsfall vor und nachteile, diese werde ich später mit messungen zeigen.
}
% Methodik und Versuchsaufbau.tex

% Definiere Variablen
% \newcommand{\Messziel}{Messziel}

% Ergebnisse und Analyse.tex

% Ergebnisse und Analyse Diagramme.tex

\newcommand{\MergesortMesswerte}{%
    \addplot[blue, mark=*] coordinates {
            (25000000,3017372600)
            (50000000,6315632400)
            (100000000,12777405700)
            (200000000,27203738300)
            (400000000,56441153000)
        };
    \addlegendentry{Mergesort}
}

\newcommand{\QuicksortMesswerte}{%
    \addplot[green, mark=*] coordinates {
            (25000000,1528071600)
            (50000000,3248990900)
            (100000000,6653966800)
            (200000000,13778762100)
            (400000000,29074726600)
        };
    \addlegendentry{Quicksort}
}

\newcommand{\GrundlegendeLaufzeitenAbhaengigVonDerArraygroesseDiagrammA}{%
    \begin{tikzpicture}
        \begin{axis}[
                title style={yshift=1.5ex},
                width=0.5\textwidth,
                height=0.4\textwidth,
                xlabel={Array-Länge},
                ylabel={Dauer in ns},
                title={Länge vergrößern},
                xmin=0, xmax=5 * 10^8,
                ymin=0*10^6, ymax=6*10^10,
                grid style=dashed,
                legend pos=north west,
                ytick={1609541400,6854920900,14495472700,56441153000},
                % xtick={2^21,2^23,2^24,2^25},
                % xticklabels={$2^{21}$, $2^{23}$, $2^{24}$, $2^{25}$},
                % scaled x ticks=false,
                % scaled y ticks=false,
            ]
            \MergesortMesswerte
            \QuicksortMesswerte
            % n*log2(n)
            \addplot[black, dashed,domain=1e7:4e8, samples=100] {x*log2(x)};
            \addlegendentry{$n \cdot \log_2 n$}
            % % n
            % \addplot[red, dashed,domain=1e7:4e8, samples=100] {x};
            % \addlegendentry{$n$}
            % % log2(n)
            % \addplot[green, domain=1e7:4e8, samples=100] {log2(x)};
            % \addlegendentry{$\log_2 n$}
        \end{axis}
    \end{tikzpicture}%
}

\newcommand{\GrundlegendeLaufzeitenAbhaengigVonDerArraygroesseDiagrammB}{%
    \begin{tikzpicture}
        \begin{axis}[
                title style={yshift=1.5ex},
                width=0.5\textwidth,
                height=0.4\textwidth,
                xlabel={Array-Länge},
                ylabel={Dauer in ns},
                title={Länge vergrößern},
                xmin=0, xmax=5 * 10^8,
                ymin=0*10^6, ymax=1*10^11,
                grid style=dashed,
                legend pos=north west,
                xmode=log,
                log basis x=10,
                xtick=data,
                ymode=log,
                log basis y=10,
                ytick=data,
                % xtick={2^21,2^23,2^24,2^25},
                % xticklabels={$2^{21}$, $2^{23}$, $2^{24}$, $2^{25}$},
                % scaled x ticks=false,
                % scaled y ticks=false,
            ]
            \MergesortMesswerte
            \QuicksortMesswerte
            % n*log2(n)
            \addplot[black, dashed,domain=1e7:4e8, samples=100] {x*log2(x)};
            \addlegendentry{$n \cdot \log_2 n$}
        \end{axis}
    \end{tikzpicture}%
}


% Definiere Variablen
% \newcommand{\Messziel}{Messziel}

% % ----------------------
% % 4. Ergebnisse und Analyse
% % ----------------------
% \newpage
% \chapter{Ergebnisse und Analyse}
% \section{Grundlegende Laufzeiten abhängig von der Arraygröße}
% \subsection{Messziel} % Einleitung
% \subsection{Erwartung}
% \subsection{Diagramm}
% \subsection{Analyse und Interpretation}
% \newpage
% \section{Einfluss des Listentyps} % Listentyps: Zufall, Sortiert, InvertiertSortiert, FastSortiert, Dupliziert
% \subsection{Messziel} % Einleitung
% \subsection{Erwartung}
% \subsection{Diagramm}
% \subsection{Analyse und Interpretation}
% \newpage
% \section{Einfluss der Arraygröße im Detail}
% \subsection{Messziel} % Einleitung
% \subsection{Erwartung}
% \subsection{Diagramm}
% \subsection{Analyse und Interpretation}
% \newpage
% \section{Tiefenbasierte Thread-Erzeugung}
% \subsection{Messziel} % Einleitung
% \subsection{Erwartung}
% \subsection{Diagramm}
% \subsection{Analyse und Interpretation}
% \newpage
% \section{Workerthreads}
% \subsection{Messziel} % Einleitung
% \subsection{Erwartung}
% \subsection{Diagramm}
% \subsection{Analyse und Interpretation}
% \newpage
% \section{Vergleich der Threading-Methoden}
% \subsection{Messziel} % Einleitung
% \subsection{Erwartung}
% \subsection{Diagramm}
% \subsection{Analyse und Interpretation}
% \newpage
% \section{Einfluss des Datentyps der Liste} % Listenart: int, string
% \subsection{Messziel} % Einleitung
% \subsection{Erwartung}
% \subsection{Diagramm}
% \subsection{Analyse und Interpretation}
% % \section{Debug Vs Release}

% % ----------------------
% % 4. Ergebnisse und Analyse
% % ----------------------
% \chapter{Ergebnisse und Analyse}
% \section{Grundlegende Laufzeiten abhängig von der Arraygröße}

% \subsection{Messziel} % Einleitung
\newcommand{\GrundlegendeLaufzeitenAbhaengigVonDerArraygroesseMessziel}{
    Das Messziel besteht darin, die Abhängigkeit des seriellen Algorithmus von der Arraygröße grafisch darzustellen.
    Dadurch können diese Ergebnisse später mit den nicht-seriellen Varianten verglichen werden.
    Gleichzeitig dient dies als einfacher Einstieg in das Thema.
}

% \subsection{Erwartung}
\newcommand{\GrundlegendeLaufzeitenAbhaengigVonDerArraygroesseErwartung}{
    Da die durchschnittliche Laufzeit \(O(n \log n)\) beträgt, erwarte ich eine logarithmische Laufzeiterhöhung bei wachsender Arraygröße.
}
% \subsection{Diagramm}
\newcommand{\GrundlegendeLaufzeitenAbhaengigVonDerArraygroesseDiagramm}{
    \GrundlegendeLaufzeitenAbhaengigVonDerArraygroesseDiagrammA
    \GrundlegendeLaufzeitenAbhaengigVonDerArraygroesseDiagrammB
    \newline
    \GrundlegendeLaufzeitenAbhaengigVonDerArraygroesseDiagrammC
}

% \subsection{Analyse und Interpretation}
\newcommand{\GrundlegendeLaufzeitenAbhaengigVonDerArraygroesseAnalyse}{
    In den ersten zwei Diagrammen ist die Veränderung der Laufzeit zu sehen, wenn die Listengröße fünfmal verdoppelt wird und bei \(2.5 \cdot 10^7\) startet.
    Beim zweiten Diagramm sind die Achsen logarithmisch dargestellt, da dies die Darstellung und den Vergleich der Laufzeiten erleichtert.
    \newline
    Unter diesen beiden Diagrammen befindet sich ein drittes Diagramm, in dem die gemessenen Laufzeiten ebenfalls logarithmisch dargestellt sind und der Größenbereich von \(1\) bis \(800\,000\) betrachtet wird.
    \newline
    Anhand dieser Diagramme ist deutlich erkennbar, dass sowohl Mergesort als auch Quicksort tatsächlich eine Laufzeit von
    \(O(n \log n)\) besitzen.
    \newline
    Zudem ist erkennbar, dass Mergesort etwa doppelt so lange benötigt wie Quicksort und dass Quicksort näherungsweise eine Laufzeit von
    \(2 \cdot n \log_2(n)\) aufweist.
    Aus diesen Beobachtungen lässt sich ableiten, dass der Partitionierungsschritt eine Laufzeit von etwa \(2n\) besitzt, während der Merge-Schritt näherungsweise eine Laufzeit von \(4n\) aufweist.
    Zur Vereinfachung der Betrachtung wird jedoch bei beiden Algorithmen weiterhin von \(n\) ausgegangen.
    \newline
    Da eine lineare Laufzeit auf einen Blick leichter zu interpretieren ist, wurde zusätzlich die Funktion \(32n\) eingezeichnet.
    Anhand dieser Funktion ist erkennbar, dass sie im untersuchten Zahlenbereich von \(1\) bis \(4 \cdot 10^8\) teilweise sogar eine genauere Abschätzung liefert als \(n \log_2(n)\).
    Daraus folgt, dass von einer begründeten Mindestlaufzeit von \(32n\) ausgegangen werden kann.
    \newline
    Abschließend ist anzumerken, dass alle Messungen mit einer Laufzeit von kleiner oder gleich \(10^4\,\text{ns}\) aufgrund der Messtoleranz nur eine eingeschränkte Aussagekraft besitzen.
    Zwar kann mit \texttt{chrono} auf eine Auflösung von \(100\,\text{ns}\) genau gemessen werden, dennoch verbleiben natürliche Schwankungen, die insbesondere im Bereich von \(10^4\,\text{ns}\) einen erheblichen Einfluss auf die Messergebnisse haben.
}
% Diskussion und Fazit.tex

% Definiere Variablen
% \newcommand{\Messziel}{Messziel}
\newcommand{\InterpretationAllerErgebnisse}{
    Es ist eindeutig zu sehen, dass Mergesort in der tiefenbasierten Thread-Erzeugungsvariante am besten performt sowie dass Quicksort bei der Worker-Thread-Variante nach einem Work-Stealing-Ansatz die beste Laufzeit erzielt, exakt wie zu erwarten war.
    Es ist auch sehr eindeutig, dass diese Varianten nur leicht schlechtere Laufzeiten erreichen, als theoretisch möglich wäre.
    Ich hatte allerdings ebenfalls erwartet, dass die Ergebnisse aufgrund der anfangs genannten Overheads stärker von der theoretisch möglichen Laufzeit abweichen würden.
    Wenn man nun die entstehenden Overheads durch die Thread-Erzeugung mit den seriellen Laufzeiten vergleicht, erhält man einen guten Eindruck, ab wann Parallelisierung überhaupt einen Vorteil bietet.
    Um zum Beispiel 16 Threads effizient nutzen zu können, sollte eine Mindestlaufzeit von 1\,ms erreicht werden. Dies ist z.\,B. bei Quicksort der Fall, wenn ein \texttt{int}-Array die Mindestgröße von 20k hat.
    Zusammenfassend lässt sich festhalten, dass Mergesort insbesondere dann die bessere Wahl ist, wenn die \textit{Worst-Case}-Laufzeit von Quicksort nicht tolerierbar ist. Allerdings zeigt Quicksort für typische, nicht degenerierte Listen im Durchschnitt eine bessere Laufzeit, wobei die \textit{Best-} und \textit{Worst-Case}-Laufzeiten stets berücksichtigt werden sollten.
}
%\section{Ausblick und weiterführende Überlegungen}
% Ausblick und weiterführende Überlegungen
\newcommand{\Ausblick}{
    Eine theoretische Variante könnte Mergesort als Grundalgorithmus verwenden und bei Teilarrays kleiner gleich 40k auf Quicksort wechseln. Dadurch lässt sich eine absehbare Worst-Case-Laufzeit gewährleisten, während die durchschnittliche Laufzeit gegenüber reinem Mergesort verbessert wird.

    Man kann ebenso die Worst-Case-Laufzeit von Quicksort unwahrscheinlicher machen. Ein Ansatz dafür ist, dass das Pivot-Element zufällig bestimmt wird.
    Man kann auch mehrere Elemente zufällig auswählen sowie die Listenränder und die Listenmitte in die Pivot-Auswahl einbeziehen und daraus den Median bilden.
    Die Anzahl der zufällig hinzugewählten Elemente könnte dabei abhängig von der Listengröße gewählt werden.
    Diese Methoden führen dazu, dass die Worst-Case-Laufzeit nur noch eine Frage der Wahrscheinlichkeit ist.
    Da die Wahrscheinlichkeit sinkt, je mehr Pivot-Elemente zufällig bestimmt werden, ist es in der Praxis dann extrem unwahrscheinlich, dass der Worst-Case jemals eintritt.

    Es gibt auch Varianten, die den Partitionierungs- und Merge-Schritt parallelisiert haben. Man kann diese Variante auch noch mit der rekursiv parallelisierten Variante kombinieren, um so immer von allen Threads profitieren zu können. Dies sorgt dann wieder für eine noch bessere Laufzeit.

    Ich möchte noch erwähnen, dass moderne CPUs auf einem Multi-Chipsatz-Design basieren.
    Bei Intel sind das alle CPUs mit Performance- und Effizienz-Kernen zugleich.
    Da diese Kerne unterschiedliche Taktraten und Leistungen erbringen können, erschwert dies die perfekte Parallelisierung.
    Bei AMD besteht zum Beispiel der 7950X3D aus einem 7800X3D und einem 7800X.
    Diese beiden Chips unterscheiden sich in der Cache-Größe, in der Cache-Latenz und in der Taktrate.
    Der 7800X3D überhitzt auch architekturbedingt leichter, da der 3D-Cache direkt auf den Kernen sitzt.
    Dies sorgt auch bei AMD dafür, dass sich diese CPU nur sehr schwer perfekt parallelisieren lässt.
    Aus diesen Gründen sowie aufgrund der anfangs genannten Overheads halte ich eine perfekte Parallelisierung moderner CPUs für unmöglich. Stattdessen kann man sich dem idealen Szenario nur annähern.
}

% Auf Paralel merge und PArtionire eingehen.

% Aufgrund der weiterhin bestehenden Worst-Case-Komplexität von Quicksort ist für sehr große Arrays die Verwendung von Mergesort eindeutig zu bevorzugen.
% \newline
% Für typische, nicht degenerierte Listen zeigt Quicksort im Durchschnitt eine konstante und effiziente Leistungssteigerung, wobei die Best- und Worst-Case-Laufzeiten stets berücksichtigt werden sollten.
% Anhang.tex

% Definiere Variablen
% \newcommand{\Messziel}{Messziel}


\newcommand{\HardwareSpezifikationen}{
    % Hardware genau dokumentieren
    Zur besseren Einordnung der Leistungsfähigkeit der verwendeten Hardware befindet sich unter folgendem Link ein Benchmark:
    \newline
    \href{https://www.userbenchmark.com/UserRun/70984567}{https://www.userbenchmark.com/UserRun/70984567}
    \newline
    Ich liste aber jetzt hier auch nochmal die relevanten Hardwarekomponenten auf.
    \newline
    \href{https://geizhals.de/amd-ryzen-7-5800x-100-100000063wof-a2392525.html?t=alle&plz=&va=b&vl=de&hloc=de&v=e&togglecountry=set}
    {CPU: AMD Ryzen 7 5800X, 8C/16T, 3.80-4.70GHz}
    \newline
    \href{https://geizhals.de/be-quiet-dark-rock-pro-4-bk022-a1794846.html?hloc=de}
    {CPU Kühler: be quiet! Dark Rock Pro 4)}
    \newline
    \href{https://geizhals.de/g-skill-aegis-dimm-kit-16gb-f4-3200c16d-16gis-a2151626.html?hloc=de}
    {RAM: G.Skill Aegis UDIMM 16GB Kit, DDR4-3200, CL16-18-18-38 (2 Kits, insgesamt 4x8 GB = 32 GB)}
    \newline
    Betriebssystem: Windows 10 Version 22H2
}

\newcommand{\Code}{
    % Link oder Verweis auf Repository
    Der gesamte Quellcode dieser Arbeit ist öffentlich unter folgendem Link verfügbar:
    \newline
    \href{https://github.com/Leon333M/Sortierverfahren}{https://github.com/Leon333M/Sortierverfahren}
}

\newcommand{\Quellen}{
    Die verwendeten Quellen dienen ausschließlich der Einordnung etablierter
    Algorithmen und Konzepte. Alle Laufzeitanalysen und Herleitungen wurden
    eigenständig durchgeführt.
}
% https://de.wikipedia.org/wiki/Amdahlsches_Gesetz
% https://de.wikipedia.org/wiki/Quicksort
% https://de.wikipedia.org/wiki/Parallel_Quicksort
% https://de.wikipedia.org/wiki/Mergesort
% \begin{thebibliography}{9}

%     \bibitem{amdahl}
%     Wikipedia:
%     \textit{Amdahlsches Gesetz},
%     \url{https://de.wikipedia.org/wiki/Amdahlsches_Gesetz},
%     abgerufen am 01.01.2026.

%     \bibitem{quicksort}
%     Wikipedia:
%     \textit{Quicksort},
%     \url{https://de.wikipedia.org/wiki/Quicksort},
%     abgerufen am 01.01.2026.

%     \bibitem{parallelquicksort}
%     Wikipedia:
%     \textit{Parallel Quicksort},
%     \url{https://de.wikipedia.org/wiki/Parallel_Quicksort},
%     abgerufen am 01.01.2026.

%     \bibitem{mergesort}
%     Wikipedia:
%     \textit{Mergesort},
%     \url{https://de.wikipedia.org/wiki/Mergesort},
%     abgerufen am 01.01.2026.

% \end{thebibliography}
% Absicherung eigenständiger Arbeit

\begin{document}

\begin{frame}
    \titlepage
\end{frame}

\begin{frame}{Agenda}
    \tableofcontents
\end{frame}

\section{Einleitung}
\begin{frame}{Einleitung: Motivation}
    \textbf{Motivation:}
    \Motivation
\end{frame}

\begin{frame}{Einleitung: Zielsetzung und Forschungsfrage}
    \textbf{Zielsetzung und Forschungsfrage:}
    \ZielsetzungUndForschungsfrage
\end{frame}

\section{Theoretische Grundlagen}
\begin{frame}{Theoretische Grundlagen}
    \textbf{Hinweis zur mathematischen Darstellung}
    Die in dieser Arbeit genutzten mathematischen Beschreibungen und Formeln beziehen sich durchgehend auf die konkret implementierten Programmstrukturen und können daher von allgemeinen Standardformeln abweichen.
\end{frame}

\begin{frame}{Sortieralgorithmen: Quicksort und Mergesort}
    Sowohl \textbf{Quicksort} als auch \textbf{Mergesort} basieren auf dem \emph{Teile-und-Herrsche}-Prinzip und sind rekursive Sortieralgorithmen. Dabei wird das zu sortierende Array wiederholt in kleinere Teilprobleme zerlegt, die unabhängig voneinander verarbeitet werden.
\end{frame}

\begin{frame}{Sortieralgorithmen: Mergesort}
    Das Grundprinzip von \textbf{Mergesort} besteht darin, zwei bereits sortierte Teilarrays zu einem sortierten Array zusammenzuführen. In dieser Arbeit wird das unsortierte Eingabearray rekursiv in zwei möglichst gleich große Hälften geteilt, bis jedes Teilarray nur noch aus einem einzelnen Element besteht. Da ein Array mit einem Element per Definition sortiert ist, beginnt anschließend der sogenannte \emph{Merge-Schritt (Mischen)}. In diesem Schritt werden jeweils zwei sortierte Teilarrays zu einem sortierten Gesamtergebnis zusammengeführt.
    \newline
    Hierfür werden beide Teilarrays mit einer Gesamtlänge von \(n\) Elementen sequenziell durchlaufen und die Elemente verglichen.
    Der Aufwand pro Merge-Schritt entspricht dabei \(n\) Vergleichen, da jedes Element genau einmal betrachtet wird, sowie \(2n\) Lese- und Schreibzugriffen, da die Elemente temporär in ein neues Array der Größe \(n\) geschrieben und von dort wieder gelesen werden müssen.
\end{frame}

\begin{frame}[fragile]{Sortieralgorithmen: Mergesort}

    \begin{lstlisting}[language=C++, basicstyle=\ttfamily\small]
    void Mergesort::mergesort(int *liste, const int links, const int rechts) {
        int laenge = rechts - links + 1;
        if (laenge > 1) {
            int mitte = links + ((rechts - links) / 2);
            mergesort(liste, links, mitte);      // A
            mergesort(liste, mitte + 1, rechts); // B
            mischen(liste, links, mitte, rechts, laenge);
        }
    }
    \end{lstlisting}

\end{frame}

\begin{frame}[fragile]{Sortieralgorithmen: Mergesort}

    \begin{lstlisting}[language=C++, basicstyle=\ttfamily\small]
    void Mergesort::mischen(int *liste, int links, const int mitte, const int rechts, const int lange) {
        int *listeB = new int[lange];
    
        // Kopiere nach listeB
        for (int i = links; i < mitte + 1; i++) {
            listeB[i - links] = liste[i];
        }
        for (int i = mitte + 1; i < rechts + 1; i++) {
            listeB[lange - 1 + mitte + 1 - i] = liste[i];
        }
    
        // Sortiere liste
        int i = 0;         // links
        int j = lange - 1; // rechts
        int k = links;     // links
        while (i < j) {
            if (listeB[i] < listeB[j]) {
                liste[k] = listeB[i];
                i++;
            } else {
                liste[k] = listeB[j];
                j--;
            }
            k++;
        }
        liste[rechts] = listeB[i];
    
        delete[] listeB;
    };    
    \end{lstlisting}

\end{frame}

\begin{frame}[fragile]{Sortieralgorithmen: Mergesort}

    \begin{lstlisting}[language=C++, basicstyle=\ttfamily\small]
        ...
        // Sortiere liste
        int i = 0;         // links
        int j = lange - 1; // rechts
        int k = links;     // links
        while (i < j) {
            if (listeB[i] < listeB[j]) {
                liste[k] = listeB[i];
                i++;
            } else {
                liste[k] = listeB[j];
                j--;
            }
            k++;
        }
        liste[rechts] = listeB[i];
        delete[] listeB;
    };    
    \end{lstlisting}

\end{frame}

\begin{frame}{Sortieralgorithmen: Mergesort}
    \begin{flalign*}
        T(n)    & = m_1 + m_2 + n                                                                                                             & \\
        T(n)    & = T \left ( \left \lfloor \frac{n}{2} \right \rfloor \right) + T \left ( \left \lceil \frac{n}{2} \right \rceil \right) + n & \\
        T(n)    & = 2 \cdot T \left( \frac{n}{2} \right) + n,                                                                                 & \\
        T(n)    & = n \cdot \log_2(n) + n,                                                                                                    & \\
        O(T(n)) & = O(n \log n).                                                                                                              &
    \end{flalign*}
\end{frame}


\begin{frame}{Sortieralgorithmen: Mergesort}
    Balancierter Binärbaum für $n = 16$
    \newline
    \newline
    \BalancierterBinaerbaumNSechzehn
\end{frame}

\begin{frame}{Sortieralgorithmen: Quicksort}
    \textbf{Quicksort} ist im Grundaufbau ähnlich strukturiert, unterscheidet sich jedoch wesentlich im Ablauf. Die Liste wird nicht zwingend in zwei gleich große Hälften geteilt. Stattdessen wird zunächst ein sogenanntes \emph{Pivot-Element} gewählt, anhand dessen die Liste in einen kleineren und einen größeren Teil partitioniert wird. Dieser Partitionierungsschritt erfolgt vor den rekursiven Selbstaufrufen.
    Beim Partitionieren wird die Liste so umsortiert, dass alle Elemente, die kleiner als das Pivotelement sind, links davon stehen und alle Elemente, die größer sind, rechts davon stehen. Dabei werden die Elemente auf beiden Seiten entsprechend getauscht.
\end{frame}

\begin{frame}[fragile]{Sortieralgorithmen: Quicksort}

    \begin{lstlisting}[language=C++, basicstyle=\ttfamily\small]
    void Quicksort::quicksort(int *liste, const int links, const int rechts) {
        if (links < rechts) {
            int ml, mr;
            partitioniere(liste, links, rechts, ml, mr);
            quicksort(liste, links, ml);
            quicksort(liste, mr, rechts);
        }
    };
    \end{lstlisting}

\end{frame}

\begin{frame}[fragile]{Sortieralgorithmen: Quicksort}

    \begin{lstlisting}[language=C++, basicstyle=\ttfamily\small]
    void Quicksort::partitioniere(int *liste, const int links, const int rechts, int &ml, int &mr) {
        int i = links;
        int j = rechts;
        int mitte = links + ((rechts - links) / 2);
        int p = liste[mitte];
        while (i <= j) {
            while (liste[i] < p) {
                i++;
            }
            while (liste[j] > p) {
                j--;
            }
            if (i <= j) {
                vertausche(liste, i, j);
                i++;
                j--;
            }
        };
        ml = j; mr = i;
    };
    \end{lstlisting}

\end{frame}

\begin{frame}[fragile]{Sortieralgorithmen: Quicksort}

    \begin{lstlisting}[language=C++, basicstyle=\ttfamily\small]
    void Quicksort::vertausche(int *liste, const int a, const int b) {
        int temp = liste[a];
        liste[a] = liste[b];
        liste[b] = temp;
    };
    \end{lstlisting}

\end{frame}

\begin{frame}{Sortieralgorithmen: Quicksort Best-Case}
    \textbf{Best-Case} von Quicksort:
    \begin{flalign*}
        T(n)    & = q_1 + q_2 + n                                                                                                             & \\
        T(n)    & = T \left ( \left \lfloor \frac{n}{2} \right \rfloor \right) + T \left ( \left \lceil \frac{n}{2} \right \rceil \right) + n & \\
        T(n)    & = 2 \cdot T \left( \frac{n}{2} \right) + n,                                                                                 & \\
        T(n)    & = n \cdot \log_2(n) + n,                                                                                                    & \\
        O(T(n)) & = O(n \log n).                                                                                                              &
    \end{flalign*}
\end{frame}

\begin{frame}{Sortieralgorithmen: Quicksort Best-Case}
    Balancierter Binärbaum für $n = 16$
    \newline
    \newline
    \BalancierterBinaerbaumNSechzehn
\end{frame}

\begin{frame}{Sortieralgorithmen: Quicksort}
    Der \textbf{Worst-Case} von Quicksort ist:
    \begin{flalign*}
        T(n)    & = q_1 + q_2 + n                      & \\
        T(n)    & = T(n-1) + 1 + n,                    & \\
        T(n)    & = \frac{1}{2} \cdot ( n^2 + n ) + n, & \\
        O(T(n)) & = O(n^2).                            &
    \end{flalign*}

    Der \textbf{heuristisch betrachtete Average-Case} von Quicksort ist:
    \begin{flalign*}
        T(n)    & = q_1 + q_2 + n                                                                                                                                 & \\
        q_1     & = T\left( \frac{1 + \dots + (n-1)}{n-1} \right) = T\left( n \cdot \frac{n-1}{2} \cdot \frac{1}{n-1} \right) = T\left( \frac{n}{2} \right) = q_2 & \\
        T(n)    & = 2 \cdot T\left( \frac{n}{2} \right) + n                                                                                                       & \\
        T(n)    & = n \cdot \log_2(n) + n                                                                                                                         & \\
        O(T(n)) & = O(n \log n).                                                                                                                                  &
    \end{flalign*}
\end{frame}

\begin{frame}{Sortieralgorithmen: Quicksort}
    Maximal unbalancierter Binärbaum für $n = 4$ \textbf{Worst-Case}:
    \newline
    \newline
    \QuicksortWorstCaseBaum
\end{frame}

\begin{frame}{Grundlagen der Parallelisierung}
    \begin{itemize}
        \item \textbf{Amdahlsches Gesetz}, Speedup $1/p$
        \item theoretischer Speedup aufgrund von Overheads nie erreichbar
    \end{itemize}
    \begin{flalign*}
        p       & = \text{Thread-Anzahl} ,                                                                                                   & \\
        f       & = \text{serieller Code, wobei } 0 < f \le 1 ,                                                                              & \\
        t ( p ) & = \underbrace{f \cdot t(1)}_{\text{serielle Arbeit}} + \underbrace{(1-f) \cdot \frac{t(1)}{p}}_{\text{parallele Arbeit}} . &
    \end{flalign*}
    $f$ nahe 0 bedeutet,
    dass der Code fast die ganze Zeit alle Recheneinheiten beschäftigt hält.
    Dies impliziert, dass sein Nebenläufigkeitsgrad immer gleich oder höher als die Anzahl der parallelen
    Rechenknoten ist.
    $f$ nahe 1 bedeutet, dass der Großteil unseres Codes nicht parallel läuft.
    \cite{Weinzierl2024}
    \begin{itemize}
        \item Es ist kein linearer Speedup ($1/p$), daher nur eine eigene $T(n,p)$-Formel statt des exakten Amdahlschen Gesetzes.
    \end{itemize}
\end{frame}

\begin{frame}{Thread-Modelle, Overheads und Skalierungsgrenzen}
    \begin{itemize}
        \item Overheads
        \item Skalierungsgrenzen
        \item Thread-Modelle, Implementierungsstrategien
    \end{itemize}
\end{frame}

\begin{frame}{Overheads}
    \begin{itemize}
        \item Betrachten \textbf{Overheads} aus siecht eines Softwarearchitekten, für den die gesamte Hardware eine Blackbox ist.
        \item \textbf{Overheads} (Zusatzlaufzeiten) verhindern theoretisch ideale Zeitersparnis.
        \item \textbf{Overheads}:
              \begin{itemize}
                  \item Initialisierungs- und Join-Zeiten
                  \item Destruktoren
                  \item Synchronisation
                  \item Swapping (Kontextwechsel)
                  \item Speicherlatenzen (begrenzte Speichergröße, Cache-Misses, Datenübertragungsgeschwindigkeit (Latenzen), Datenübertragungsrate)
                  \item ...
              \end{itemize}
    \end{itemize}
\end{frame}

\begin{frame}{Overheads}
    In der Praxis verursachen Threads verschiedene \textbf{Overheads} (Zusatzlaufzeiten), die verhindern, dass eine theoretisch ideale Zeitersparnis erreicht wird. Zu diesen Overheads zählen primär die Initialisierungs- und Join-Zeiten, die Laufzeiten von Destruktoren sowie die notwendige Synchronisation bei Abhängigkeiten zwischen Threads. Um die Datenkonsistenz zu gewährleisten, müssen Mechanismen wie Sperren (Mutexe) oder Barrieren (Synchronisationspunkte) eingesetzt werden, welche zusätzliche Wartezeiten und Verwaltungsoverheads verursachen. Parallel dazu setzen Hardware-Limitierungen der Skalierung Grenzen. Hierbei beeinflussen Context-Switching-Zeiten bei Überbelegung der Kerne (Oversubscription), die begrenzte Anzahl physischer Kerne (im Testsystem 8 physische bzw. 16 logische Prozessoren) sowie eine erhöhte Rate an Cache-Misses bei steigender Thread-Anzahl die Performance negativ. Letzteres führt dazu, dass vermehrt Daten aus dem RAM geladen werden müssen, wodurch das System je nach Anwendungsfall eher durch die Bandbreite und Speicherlatenz des Speichercontrollers (Memory Bound) als durch die Rechenleistung der CPU-Kerne begrenzt wird. Zudem ist zwischen physischen und logischen Prozessoren zu unterscheiden, da letztere aufgrund geteilter Hardware-Ressourcen weniger effizient skalieren.
\end{frame}

\begin{frame}{Skalierungsgrenzen: Hardware-Grenze CPU}
    \begin{itemize}
        \item Begrenzte Anzahl physischer Kerne (im Testsystem 8 physische bzw. 16 logische Prozessoren).
        \item Je größer die Leistungsaufnahme der CPU ist, desto ineffizienter arbeitet sie.
        \item Begrenzte maximale Leistungsaufnahme (Thermal Design Power, TDP), welche dafür sorgt, dass mehr Threads nicht unbedingt mehr Leistung bedeuten.
        \item Leistungsaufnahme wird größtenteils zu Abwärme, die im schlimmsten Fall zur Runtertaktung (Thermal Throttling) einzelner CPU-Kerne führt.
        \item Der wenige Rest der Leistungsaufnahme baut ein sich ständig wechselndes Magnetfeld auf, welches dabei Störströme (Kreisströme) induziert.
    \end{itemize}
\end{frame}

\begin{frame}{Overheads: Hardware-Grenze CPU}
    Eine weitere wesentliche Hardware-Grenze stellt die CPU dar. Moderne CPUs sind durch eine maximale Leistungsaufnahme (Thermal Design Power, TDP) begrenzt. Eine höhere Auslastung aller Kerne führt daher nicht zwangsläufig zu proportional höherer Leistung. Beispielsweise weist die genutzte CPU einen Single-Core-Boost-Takt von 4,75 GHz auf, jedoch nur einen All-Core-Takt von 4,6 GHz, wodurch einzelne Threads bei geringer Auslastung performanter laufen. Zudem wird ein Großteil der TDP als Abwärme freigegeben, die effizient abgeführt werden muss. Bei unzureichender Kühlung oder Überschreitung thermischer Grenzen kommt es zur automatischen thermischen Drosselung (Thermal Throttling) der CPU, wodurch der Takt des jeweiligen Kerns temporär reduziert wird.
    Der wenige Rest der TDP wird in elektromagnetische Felder umgewandelt, welche dann für Störströme (Kreisströme) sorgen.
    Diese Faktoren beeinflussen die Performance ebenfalls negativ.
\end{frame}

\begin{frame}{Thread-Modelle, Implementierungsstrategien}
    \begin{itemize}
        \item Jeden Codeabschnitt ohne sequentielle Abhängigkeiten in einen neuen Thread auslagern
        \item Thread-Anzahl begrenzen, da sonst oft Overheads dominieren.
        \item Nutzen von Worker-Threads (Threadpool)
        \item Work-Stealing-Ansatz
        \item Nur neue Aufgaben auf Aufgabenliste legen, wenn freie Worker-Threads vorhanden sind.
    \end{itemize}
\end{frame}

\begin{frame}{Thread-Modelle, Implementierungsstrategien}
    Hinsichtlich der Implementierung existieren verschiedene Ansätze. Die simpelste Methode besteht darin, jeden Codeabschnitt ohne sequentielle Abhängigkeiten in einen neuen Thread auszulagern. Dies ist jedoch oft kontraproduktiv, da ein Übermaß an Threads zu Performance-Verlusten durch Context Switching und hohen Speicherverbrauch führt. In der Praxis wird die Thread-Anzahl daher meist limitiert.
    \newline
    Eine Optimierung stellt die Nutzung von \textbf{Worker-Threads} dar. Hierbei werden Threads einmalig initialisiert und verbleiben über die gesamte Laufzeit aktiv, um kontinuierlich neue Aufgaben abzuarbeiten, anstatt nach jeder Aufgabe zerstört zu werden. Eine weiterführende Strategie ist das Dynamic Scheduling (oder Work-Stealing-Ansätze), bei dem Aufgaben nur dann zugewiesen werden, wenn Ressourcen frei sind. Sind alle Worker-Threads belegt, kann der aufrufende Thread die Aufgabe selbst bearbeiten, um Wartezeiten zu minimieren. Die Vor- und Nachteile dieser Strategien werden im Abschnitt der Implementierungsvarianten und der Messungen detailliert analysiert.
\end{frame}

\section{Methodik und Versuchsaufbau}
\begin{frame}{Messumgebung und Implementierung}
    % Hier Details aus \MessumgebungUndHardware
\end{frame}

\begin{frame}[fragile]{Implementierungsvarianten Code}
    % [fragile] ist nötig für Listings auf Folien
\end{frame}

\section{Ergebnisse und Analyse}
\begin{frame}{Laufzeitmessungen (sequenziell)}
    % Platz für Diagramme aus \GrundlegendeLaufzeitenAbhaengigVonDerArraygroesseDiagramm
\end{frame}

\begin{frame}{Analyse der Threading-Methoden}
    % Vergleich Workerthreads vs. Tiefenbasierte Erzeugung
\end{frame}

\section{Diskussion und Fazit}
\begin{frame}{Beantwortung der Forschungsfrage}
    % Kernpunkte aus \BeantwortungDerForschungsfrage
\end{frame}

\begin{frame}{Zusammenfassung und Ausblick}
    % Kernpunkte aus \Zusammenfassung und \Ausblick
\end{frame}

\begin{frame}
    \centering
    \Huge Vielen Dank für Ihre Aufmerksamkeit!\\
    \vspace{1cm}
    \large Fragen und Diskussion
\end{frame}

\alleFolien

\begin{frame}[allowframebreaks]{Literatur}
    \bibliographystyle{plain}
    \bibliography{literatur}
\end{frame}

\end{document}
