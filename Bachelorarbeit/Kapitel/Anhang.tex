% Anhang.tex

% Definiere Variablen
% \newcommand{\Messziel}{Messziel}


\newcommand{\HardwareSpezifikationen}{
    % Hardware genau dokumentieren
    Zur besseren Einordnung der Leistungsfähigkeit der verwendeten Hardware befindet sich unter folgendem Link ein Benchmark:
    \newline
    \href{https://www.userbenchmark.com/UserRun/70984567}{https://www.userbenchmark.com/UserRun/70984567}
    \newline
    Ich liste aber jetzt hier auch nochmal die relevanten Hardwarekomponenten auf.
    \newline
    \href{https://geizhals.de/amd-ryzen-7-5800x-100-100000063wof-a2392525.html?t=alle&plz=&va=b&vl=de&hloc=de&v=e&togglecountry=set}
    {CPU: AMD Ryzen 7 5800X, 8C/16T, 3.80-4.70GHz}
    \newline
    \href{https://geizhals.de/be-quiet-dark-rock-pro-4-bk022-a1794846.html?hloc=de}
    {CPU Kühler: be quiet! Dark Rock Pro 4)}
    \newline
    \href{https://geizhals.de/g-skill-aegis-dimm-kit-16gb-f4-3200c16d-16gis-a2151626.html?hloc=de}
    {RAM: G.Skill Aegis UDIMM 16GB Kit, DDR4-3200, CL16-18-18-38 (2 Kits, insgesamt 4x8 GB = 32 GB)}
    \newline
    Betriebssystem: Windows 10 Version 22H2
}

\newcommand{\Code}{
    % Link oder Verweis auf Repository
    Der gesamte Quellcode dieser Arbeit ist öffentlich unter folgendem Link verfügbar:
    \newline
    \href{https://github.com/Leon333M/Sortierverfahren}{https://github.com/Leon333M/Sortierverfahren}
}

\newcommand{\Quellen}{
    Die verwendeten Quellen dienen ausschließlich der Einordnung etablierter
    Algorithmen und Konzepte. Alle Laufzeitanalysen und Herleitungen wurden
    eigenständig durchgeführt.
}
% https://de.wikipedia.org/wiki/Amdahlsches_Gesetz
% https://de.wikipedia.org/wiki/Quicksort
% https://de.wikipedia.org/wiki/Parallel_Quicksort
% https://de.wikipedia.org/wiki/Mergesort
% \begin{thebibliography}{9}

%     \bibitem{amdahl}
%     Wikipedia:
%     \textit{Amdahlsches Gesetz},
%     \url{https://de.wikipedia.org/wiki/Amdahlsches_Gesetz},
%     abgerufen am 01.01.2026.

%     \bibitem{quicksort}
%     Wikipedia:
%     \textit{Quicksort},
%     \url{https://de.wikipedia.org/wiki/Quicksort},
%     abgerufen am 01.01.2026.

%     \bibitem{parallelquicksort}
%     Wikipedia:
%     \textit{Parallel Quicksort},
%     \url{https://de.wikipedia.org/wiki/Parallel_Quicksort},
%     abgerufen am 01.01.2026.

%     \bibitem{mergesort}
%     Wikipedia:
%     \textit{Mergesort},
%     \url{https://de.wikipedia.org/wiki/Mergesort},
%     abgerufen am 01.01.2026.

% \end{thebibliography}
% Absicherung eigenständiger Arbeit