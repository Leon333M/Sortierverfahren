% Methodik und Versuchsaufbau.tex

% Definiere Variablen
% \newcommand{\Messziel}{Messziel}

% % ----------------------
% % 3. Methodik und Versuchsaufbau
% % ----------------------
% \newpage
% \chapter{Methodik und Versuchsaufbau}
% \section{Messumgebung und Hardware}
% % CPU, RAM, OS, Compiler
% \section{Implementierungsvarianten}
% % Sequenziell, parallel, Workerthreads, neue Threads
% \section{Messmethodik}
% % Vorgehensweise zur Laufzeitmessung, Datenerhebung

% \section{Messumgebung und Hardware}
% % CPU, RAM, OS, Compiler
\newcommand{\MessumgebungUndHardware}{
    Die relevaten Hardwekompenteten werden im anhang aufgelsitet. Die kurzfasung dafon ist das das test sytem
    auf ein 8 Kerner Cpu duschgefürht wurden (8 physische bzw. 16 logische Prozessoren). Mit 32 GB Ram.
    Auf den betibsytem Windows 10 Version 22H2.
    Der code ist in der sprache c++ geschreiben und über Camke mit MSVC Compeliert wurden. Die Test wurden in intigriten Terminal von VS-code als Relise Version Ausgeführt. Für den fall das dis nicht der fall war gebe ich das immer noch mal extra an.
}
