% Methodik und Versuchsaufbau.tex

% Definiere Variablen
% \newcommand{\Messziel}{Messziel}

% % ----------------------
% % 3. Methodik und Versuchsaufbau
% % ----------------------
% \newpage
% \chapter{Methodik und Versuchsaufbau}
% \section{Messumgebung und Hardware}
% % CPU, RAM, OS, Compiler
% \section{Implementierungsvarianten}
% % Sequenziell, parallel, Workerthreads, neue Threads
% \section{Messmethodik}
% % Vorgehensweise zur Laufzeitmessung, Datenerhebung

% \section{Messumgebung und Hardware}
% % CPU, RAM, OS, Compiler
\newcommand{\MessumgebungUndHardware}{
    Die relevanten Hardwarekomponenten werden im Anhang detailliert aufgelistet. Zusammenfassend wurden die Tests auf einem System mit einer 8-Kern-CPU durchgeführt (8 physische Kerne bzw. 16 logische Prozessoren) sowie mit 32~GB Arbeitsspeicher.
    \newline
    Als Betriebssystem kam Windows~10 in der Version~22H2 zum Einsatz. Der Code ist in der Programmiersprache \texttt{C++} implementiert und wurde mittels CMake mit dem MSVC-Compiler kompiliert. Die Ausführung der Tests erfolgte im integrierten Terminal von Visual Studio Code in einer Release-Konfiguration. Sollte hiervon abgewichen worden sein, wird dies an entsprechender Stelle gesondert angegeben.

}
