\newcommand{\WorkerThreadErklaerung}{
    Ein \emph{Worker-Thread} ist ein Thread, der einmalig erzeugt wird und über die gesamte Laufzeit des Programms aktiv bleibt.
    Seine Hauptaufgabe besteht darin, Aufgaben aus einer zugewiesenen Warteschlange kontinuierlich abzuarbeiten.
    Dies reduziert den Overhead durch ständiges Erzeugen und Zerstören von Threads und ermöglicht eine effiziente parallele Verarbeitung.
}

\newcommand{\WorkStealingErklaerung}{
    \emph{Work-Stealing} bezeichnet ein Verfahren, bei dem Aufgaben in prozessorspezifischen Warteschlangen verteilt werden und jeder Prozessor auf seiner lokalen Warteschlange operiert.
    Dabei können Prozessoren Aufgaben aus anderen Warteschlangen stehlen, um eine gleichmäßige Lastverteilung (\emph{Load Balancing}) zu erhalten \cite{dynamicscheduling}.
}

\newcommand{\WorkStealingAehnlichErklaerung}{
    Der \emph{Work-Stealing-ähnliche Ansatz} in dieser Arbeit basiert auf einem Worker-Thread-Pool mit einer gemeinsamen Aufgabenwarteschlange.
    Alle Worker-Threads ziehen Aufgaben aus dieser Warteschlange und bearbeiten sie.
    Dazu kann jeder Worker-Thread auch neue Aufgaben in die Warteschlange legen.
    Ist die Queue leer, warten die Threads auf neue Aufgaben.
    Dadurch wird sichergestellt, dass alle Threads möglichst konstant ausgelastet sind, ohne dass für jeden Thread eine eigene Warteschlange benötigt wird.
    Dies reduziert den Overhead durch Synchronisation und vermeidet Leerlaufzeiten, ähnlich wie beim klassischen Work-Stealing.
}

\newcommand{\WorkerthreadsBegriffsdefinition}{
    \WorkerThreadErklaerung

    \WorkStealingErklaerung
}

\newcommand{\StrongScalingBegriffsdefinition}{
    Unter \emph{Strong Scaling} versteht man die Untersuchung der Laufzeit eines festen Problemumfangs bei steigender Anzahl an Recheneinheiten (Threads).
    Ziel ist es zu analysieren, wie stark sich die Laufzeit durch zusätzliche Parallelisierung verkürzt.
}

\newcommand{\WeakScalingBegriffsdefinition}{
    Unter \emph{Weak Scaling} versteht man die Untersuchung der Laufzeit, bei der der Problemumfang proportional zur Anzahl der Recheneinheiten wächst.
    Ziel ist es zu bewerten, ob die Laufzeit bei wachsender Parallelität konstant bleibt.
}

\newcommand{\TiefenbasierteThreadErzeugungBegriffsdefinition}{
    \StrongScalingBegriffsdefinition

    \WeakScalingBegriffsdefinition
}

\newcommand{\OverheadBegriffsdefinition}{
    Unter \emph{Overhead} versteht man
    sämtliche hardwarebedingten Overheads, wie z.\,B. Speicherlatenzen, sowie
    sämtliche softwarebedingten Overheads, wie z.\,B. die Synchronisation von Threads.
}