% Diskussion und Fazit.tex

% Definiere Variablen
% \newcommand{\Messziel}{Messziel}
\newcommand{\InterpretationAllerErgebnisse}{
    Eine theoretische Variante könnte Mergesort als Grundalgorithmus verwenden und bei Teilarrays kleiner gleich 40k auf Quicksort wechseln. Dadurch lässt sich eine absehbare Worst-Case-Laufzeit gewährleisten, während die durchschnittliche Laufzeit gegenüber reinem Mergesort verbessert wird.

    Man kann ebenso die Worst-Case-Laufzeit von Quicksort unwahrscheinlicher machen, indem das Pivot-Element zufällig bestimmt wird.
    Man kann auch mehrere Elemente zufällig auswählen sowie die Listenränder und die Listenmitte in die Pivot-Auswahl einbeziehen und daraus den Median bilden.
    Die Anzahl der zufällig hinzugewählten Elemente könnte dabei abhängig von der Listengröße gewählt werden.
    Diese Methoden führen dazu, dass die Worst-Case-Laufzeit nur noch eine Frage der Wahrscheinlichkeit ist.
    Da die Wahrscheinlichkeit sinkt, je mehr Pivot-Elemente zufällig bestimmt werden, ist es in der Praxis dann extrem unwahrscheinlich, dass der Worst-Case jemals eintritt.
}
