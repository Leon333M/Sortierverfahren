% Einleitung.tex

% Definiere Variablen
% \newcommand{\Messziel}{Messziel}

\newcommand{\Motivation}{
    Die Motivation dieser Arbeit war es, herauszufinden, wie sehr man mit Threads an die theoretisch erwartete Laufzeitverbesserung herankommen kann und welche Strategie dafür am besten geeignet ist.
    Da sich für diese Untersuchungen ein geeigneter, leicht verständlicher und programmierbarer Anwendungsfall anbietet, werden Sortieralgorithmen betrachtet, die sich zudem sehr gut parallelisieren lassen.
    Dazu kommt, dass ich Recherche nach Möglichkeit vermeiden wollte. Daher war es naheliegend, selbst Code zu schreiben und diesen zu analysieren, da man hierfür weniger recherchieren muss.
}

\newcommand{\ZielsetzungUndForschungsfrage}{
    % Hier Zielsetzung und Forschungsfrage einfügen
    Ziel dieser Bachelorarbeit ist die systematische Analyse der Laufzeitentwicklung paralleler Sortierverfahren. Dabei soll untersucht werden, wie sich parallele Implementierungen von Quicksort und Mergesort im Vergleich zu ihren sequentiellen Varianten verhalten.
    Im Fokus stehen insbesondere folgende Punkte:
    \begin{itemize}
        \item der Einfluss verschiedener Threadingstrategien auf die Laufzeit,
        \item die Frage, ab welcher Eingabegröße und bei welcher Anzahl von Threads ein messbarer Geschwindigkeitsvorteil entsteht,
        \item sowie die Identifikation von Thread-Management-Techniken, die für Sortieralgorithmen die besten Laufzeiten erzielen.
    \end{itemize}
    Aus diesen Aspekten ergibt sich die zentrale Forschungsfrage dieser Arbeit:
    \newline
    \textbf{Unter welchen Bedingungen liefern parallele Sortieralgorithmen anhand von Quicksort und Mergesort einen signifikanten Laufzeitvorteil gegenüber der sequentiellen Ausführung, und welche Threadingstrategien führen dabei zur besten Laufzeit?}
}
